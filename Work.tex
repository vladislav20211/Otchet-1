%% -*- coding: utf-8 -*-
\documentclass[12pt,a4paper]{scrartcl} 
\usepackage[utf8]{inputenc}
\usepackage[english,russian]{babel}
\usepackage{indentfirst}
\usepackage{misccorr}
\usepackage{graphicx}
\usepackage{amsmath}
\begin{document}
\section{Введение}
\label{sec:intro}


\begin{enumerate}
 \item Текстовая формулировка задачи
 \item Код приложения
 \item Пример формулы
 \item Скриншот программы
 \item Пример библиографических ссылок
\end{enumerate}



\section{Текстовая формулировка задачи}
\label{sec:exp}
 \ {Написать калькулятор (четыре арифметических операции с возможностью их запоминания) – аналог стандартного калькулятора Windows.}
 
\    {    Алгоритм}
\begin{enumerate}
\item 	Задается число;
\item	Вводится операция;
\item	Вводится следующее число;
\item	Так до тех пор, пока не будет введена команда очистки (например, буква с) или пока программа не завершит работу.
\end{enumerate}



\section{Ход работы}
\label{sec:exp}

\subsection{Код приложения}
\label{sec:exp:code}
\begin{verbatim}
#include<iostream>
int main() {
  setlocale( LC_ALL, "Rus");
  double da, db, dc; char op;
  std::cout << "Введите а: ";
  std::cin >> da;
  std::cout << "Введите операцию (-;+;*;/): ";
  std::cin >> op;
  std::cout << "Введите b: ";
  std::cin >> db;
  switch ( op)
  {
     case '-': dc = da - db; break;
     case '+': dc = da + db; break;
     case '*': dc = da * db; break;
     case '/': dc = da / db; break;
     default: std::cout << "Неизвестная операция: " << op; return 0;
  }
  std::cout << "Результат: " << da << op << db << " = " << dc << std::endl;
  return 0;
}
\end{verbatim}


\section{Скриншот программы}
\label{sec:intro}
\centering
\includegraphics[width=1.0\textwidth]{itog1.jpg}
\caption{1. Результат программы.}\label{fig:par}
\section{Пример библиографических ссылок}

Для изучения «внутренностей» \TeX{} необходимо 
изучить~\cite{Knuth-2003}, а для использования \LaTeX{} лучше
почитать~\cite{Lvovsky-2003, Voroncov-2005}.

\begin{thebibliography}{9}
\bibitem{Knuth-2003}Кнут Д.Э. Всё про \TeX. \newblock --- Москва: Изд. Вильямс, 2003 г. 550~с.
\bibitem{Lvovsky-2003}Львовский С.М. Набор и верстка в системе \LaTeX{}. \newblock --- 3-е издание, исправленное и дополненное, 2003 г.
\bibitem{Voroncov-2005}Воронцов К.В. \LaTeX{} в примерах. 2005 г.
\end{thebibliography}

\end{document}
